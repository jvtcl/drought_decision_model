\documentclass[11pt]{article}
\usepackage{geometry}                % See geometry.pdf to learn the layout options. There are lots.
\geometry{letterpaper}                   % ... or a4paper or a5paper or ... 
%\geometry{landscape}                % Activate for for rotated page geometry
%\usepackage[parfill]{parskip}    % Activate to begin paragraphs with an empty line rather than an indent
\usepackage{graphicx}
\usepackage{amssymb}
\usepackage{epstopdf}
\usepackage{amsmath}
\DeclareGraphicsRule{.tif}{png}{.png}{`convert #1 `dirname #1`/`basename #1 .tif`.png}

\title{Mathematical representation of the drought decision model - Shiny Version}
\author{Trisha Shrum}
%\date{}                                           % Activate to display a given date or no date

\begin{document}
\maketitle
%\section{}
%\subsection{}
 
\section{Scripts}

\subsection{global.R}
\begin{enumerate}
\item Sources other scripts
\item Javascript coding
\item Populate a new environment with rainfall gauge info: \verb!getStationGauge()!
\item Populate a new environment with constant (user) variables: \verb!getConstantVars()!
\item Setting additional variables: acres, start years, simulation lengths
\item Create state variables for practice and full runs: \verb!getSimVars()!
\item Create lists of variables for practice and full runs: \verb!practiceRuns!, \verb!simRuns!
\item Establish additional settings
\end{enumerate}  
 
\subsection{load.R}
Loads necessary packages

\subsection{shinySupport.R}
\begin{enumerate}
\item \verb!getJulyInfo! function: Calculates available and predicted forage in July, creates a
    UI to display info and allows user to select adaptation level.
    \begin{itemize}
    \item Called in \verb!simUI.R!
    \end{itemize}
\item \verb!getCowSell! function: Creates a UI for the user to select how many cows and calves to sell. Called in \verb!simUI.R!.
\item \verb!shinyInsurance! function: Calculates premium and indemnification for a specific year and
  grid cell. Currently returns are summed but this could be done on a index interval basis instead.
\end{enumerate}




 \end{document}